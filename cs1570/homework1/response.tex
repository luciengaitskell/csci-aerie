\documentclass{article}
\usepackage[utf8]{inputenc}
\usepackage{amsmath}
\usepackage{amssymb,latexsym,amsmath,epsfig,amsthm}
\usepackage{enumitem}
\usepackage{ulem}
\usepackage{algorithm}
\usepackage{algpseudocode}


\title{Homework 1}
\author{Lucien Gaitskell}
\date{October 2020}

\begin{document}

\maketitle

\section{Problem 1}

\subsection{List (with Big-O)}

Calculate Big-O of each given equation.

\begin{align}
    6n\log n &= O(n\log n) \\
    2^{100} &= O(1) \\
    \log \log n &= O(\log \log n) \\
    \log^2 n &= O(\log^2 n) \\
    2^{2^{n}} &= O(2^{2^{n}}) \\
    \lceil \sqrt{n} \rceil &= O(n^{\frac{1}{2}}) \\
    n^{0.01} &= O(n^{\frac{1}{100}}) \\
    \frac{1}{n} &= O(n^{-1}) \\
    4n^{\frac{3}{2}} &= O(n^{\frac{3}{2}}) \\
    3n^{0.5} &= O(n^{\frac{1}{2}}) \\
    5n &= O(n)\\
    \lfloor 2n \log^2n \rfloor &= O(n \log^2n) \\
    2^n &= O(2^n) \\
    n \log_4 n &= O(n \log n) \\
    4^n &= O(4^n) \\
    n^3 &= O(n^3) \\
    n^2 \log n &= O(n^2 \log n) \\
    \sqrt{\log n} &= O(\sqrt{\log n})
\end{align}

\subsection{Growth Rate Classifications}

Group each equation by their Big-O classifications.

\begin{center}
    \small From smallest to largest.
\end{center}

\begin{align*}
    \intertext{\subsubsection*{Inverse}}
    \frac{1}{n} &= O(n^{-1})
    \intertext{\subsubsection*{Constant}}
    2^{100} &= O(1)
    \intertext{\subsubsection*{Polylogarithmic}}
    \sqrt{\log n} &= O(\log^{\frac{1}{2}}n) \\
    \log^2 n &= O(\log^2 n)
    \intertext{\subsubsection*{Logarithmic}}
    \log \log n &= O(\log \log n)
    \intertext{\subsubsection*{Root}}
    3n^{0.5} &= O(n^{\frac{1}{2}}) \\
    \lceil \sqrt{n} \rceil &= O(n^{\frac{1}{2}}) \\
    n^{0.01} &= O(n^{\frac{1}{100}})
    \intertext{\subsubsection*{Linear}}
    5n &= O(n) \\
    \intertext{\subsubsection*{Linearithmic}}
    \lfloor 2n \log^2n \rfloor &= O(n \log^2n) \\
    n \log_4 n &= O(n \log n) \\
    6n\log n &= O(n\log n)
    \intertext{\subsubsection*{Quadratic}}
    4n^{\frac{3}{2}} &= O(n^{\frac{3}{2}}) \\
    n^2 \log n &= O(n^2 \log n)
    \intertext{\subsubsection*{Cubic}}
    n^3 &= O(n^3) \\
    \intertext{\subsubsection*{Exponential}}
    4^n &= O(4^n) \\
    2^n &= O(2^n) \\
    2^{2^{n}} &= O(2^{2^{n}})
\end{align*}

\subsection{Final List}

Final organization of equations based on Big-O size.

\setcounter{equation}{0}
\begin{align}
    \frac{1}{n} &= O(n^{-1}) \\
    2^{100} &= O(1) \\
    \log \log n &= O(\log \log n) \\
    \sqrt{\log n} &= O(\log^{\frac{1}{2}}n) \\
    \log^2 n &= O(\log^2 n) \\
    n^{0.01} &= O(n^{\frac{1}{100}}) \\
    3n^{0.5} =
    \lceil \sqrt{n} \rceil &= O(n^{\frac{1}{2}}) \\
    5n &= O(n) \\
    6n\log n =
    n \log_4 n &= O(n \log n) \\
    \lfloor 2n \log^2n \rfloor &= O(n \log^2n) \\
    4n^{\frac{3}{2}} &= O(n^{\frac{3}{2}}) \\
    n^2 \log n &= O(n^2 \log n) \\
    n^3 &= O(n^3) \\
    2^n &= O(2^n) \\
    4^n &= O(4^n) \\
    2^{2^{n}} &= O(2^{2^{n}})
\end{align}

\section{Problem 2}
\subsection{Question (a)}

\begin{align*}
    T(n) &= \sum_{j=1}^{n} \left(
                \sum_{k=j}^{n}\left(
                    1 + \sum_{i=j}^{k}\left(3\right)
                \right)
            \right) + 2 \\
         &= \sum_{j=1}^{n} \left(
                \sum_{k=j}^{n}\left(
                    1 + ((k-j) \cdot 3)
                \right)
            \right) + 2 \\ 
        &= \sum_{j=1}^{n} \left(
                \sum_{k=j}^{n}\left(
                    1 + 3k - 3j
                \right)
            \right) + 2 \\
        &= \sum_{j=1}^{n} \left(
                \sum_{k=j}^{n}\left(
                    3k
                \right) +
                \sum_{k=j}^{n}\left(
                    1 - 3j
                \right)
            \right) + 2 \\
        &= \sum_{j=1}^{n} \left(
                3 \sum_{k=j}^{n}
                    k
                 +
                (n-j) \cdot (1 - 3j)
            \right) + 2 \\
        &= \sum_{j=1}^{n} \left(
                3 \cdot \frac{n-j}{2} \cdot (n + j)
                 +
                (n-j) \cdot (1 - 3j)
            \right) + 2 \\
        &= \sum_{j=1}^{n} \left(
                (n-j) \cdot \left(
                    \frac{3}{2} \cdot (n + j)
                    + 1 - 3j
                \right)
            \right) + 2 \\
        &= \sum_{j=1}^{n} \left(
                (n-j) \cdot \left(
                    \frac{3}{2} n
                    - \frac{3}{2}j
                    + 1
                \right)
            \right) + 2 \\
        &= \sum_{j=1}^{n} \left(
                \frac{3}{2} n^2
                + \frac{3}{2} j^2
                - 3 nj
                + n
                - j
            \right) + 2 \\
        &= \sum_{j=1}^{n} \left(
                \frac{3}{2} n^2
                + n
            \right)
            +
            \frac{3}{2}
            \sum_{j=1}^{n}
                j^2
            -
            (3n + 1)
            \sum_{j=1}^{n}
                j
            + 2 \\
        &= n \left(
                \frac{3}{2} n^2
                + n
            \right)
            +
            \frac{3n (n+1) (2n+1)}{12}
            -
            \frac{(3n + 1) (n-1)(n+1)}{2}
            + 2 \\
        &= \left(
                \frac{3n^3}{2}
                + n ^2
            \right)
            + \left(
                \frac{n^3}{2} + \frac{3n^2}{4} + \frac{n}{4}
            \right)
            - \left(
                \frac{3n^3}{2} + \frac{n^2}{2} + \frac{3n}{2} + \frac{1}{2}
            \right)
            + 2 \\
        &= \frac{n^3}{2}
            + \frac{5n^2}{4}
            - \frac{5n}{4}
            + \frac{3}{2}
\end{align*}

\subsection{Question (b)}
\begin{align*}
    T(n) &\geq cn^3 \hspace{2mm} \text{for} \hspace{1mm} n \geq n_0\\
    T(n) = \frac{n^3}{2}
    + \frac{5n^2}{4}
    - \frac{5n}{4}
    + \frac{3}{2}
    &\geq cn^3 \\
    T^{\prime}(n) = \frac{1}{2}
    + \frac{5}{4n}
    - \frac{5}{4n^2}
    + \frac{3}{2n^3}
    &\geq c \\
    \lim_{n \to \inf} T^{\prime}(n) &= \frac{1}{2} \\
    \therefore
    T(n) &= O(n^3)
\end{align*}

\section{Problem 3}
\subsection*{Structure}

Use a \textbf{`mid'-heap} structure, where the median element is stored at the root.
This structure holds for each subsequent sub-tree, with smaller values on the left and larger on the right.

In the case where there is an even number of elements, the root node will be null.
The median operations will then yield the two medians.
Therefore, there will be a value to keep track of the number of elements.

This pasta data structure will also occupy $O(n)$ space.

\subsection*{Runtimes}
\begin{itemize}
    \item The $insert(x)$ operation will take $O(\log n)$ time due to downheap and maintain heap structure.
        In an empty heap, the insert will create the first node.
        If there are nodes, an insert will require a series of swaps at each level.
        At each level, only one of the two edges is followed and therefore will only operate $O(\log n)$ times.
        The node count value will be incremented.
    \item The $removeMedian()$ operation removes either the root node, for odd node count, or both subsequent nodes for even counts.
        This will take $O(\log n)$ time due to upheap, which is necessary to maintain the heap structure.
        A series of swaps will be required to fill the hole(s) left by the removed node(s).
        For each removed node, only one of the two edges is followed at each level and therefore will only operate $O(\log n)$ times.
        The node count value will be decremented, either once or twice accordingly.
    \item The $median()$ operation will take $O(1)$. If top node has a value, return its value.
        If there is no top node value, then return the two values of subsequent nodes.
        If there are no nodes, then return a null value.
\end{itemize}


\section{Problem 4}
\subsection{Part a}

Use a Binary Search Tree to hold each person with their given position value.
The nodes will be sorted by the position value in the line of each person.
This will make it far more efficient when looking to find the subsequent person in line.

\subsubsection{Operations}
\begin{enumerate}
    \item The $enter(p)$ operation will add an additional node to the BST.
        In order to maintain structure, this will have a runtime of $O(log n)$,
        as at each node only one side of the tree is followed to restructure.
    \item The $exit(p)$ operation will remove a node from the BST.
        In order to maintain structure, this will have a runtime of $O(log n)$,
        as at each node only one side of the tree is followed to restructure.
    \item The $throw(p)$ operation will take $O(log n)$ time.
        This is because first the person at $p$ has to be found in the BST.
        Then, finding the subsequent person can be done in constant time,
        as they are just the next node to the right.
\end{enumerate}

This structure will occupy $O(n)$ space, as one node will be created for each person.

\subsection{Part b}

Use a Balanced Binary Search Tree similar to \emph{Part (a)}.
However, the nodes will also maintain an extra value containing the maximum height in its subtree.
This addition will make it far more efficient when looking to find the subsequent person in line of at least a specific height.

\subsubsection{Operations}
\begin{enumerate}
    \item The $enter(p, h)$ operation will be nearly identical to above.
        However, an additional step will be added at each traversed node,
        to update the maximum height of each subtree, moving up and down the tree.
        This is a constant time addition, and therefore will also have a runtime of $O(log n)$.
    \item The $exit(p)$ operation will be nearly identical to above.
        However, an additional step will be added at each traversed node,
        to update the maximum height of each subtree, moving up and down the tree.
        This is also a constant time addition, and therefore will have a runtime of $O(log n)$.
    \item The $throw(p, t)$ operation will be similar to that above.
        However, after finding finding the person who throws the tomato,
        the search will have to traverse up and back down the tree.
        The initial traverse up and to the right, will be to find the closest subtree
        with a maximum hight $\geq t$.
        Subsequently, the tree will then be traversed back down, initially trying the left
        and subsequently right nodes, based on an acceptable max height value.
        Once a leaf is found, or the children have max height values $< t$, then that person
        is returned as the recipient of the tomato.
        As these vertical traversals are done sequentially, and only along one edge of the B-BST,
        the addition to runtime is $O(log n)$, which resolves overall to $O(log n)$.
\end{enumerate}

This structure is still a B-BST, with one extra value per node.
As a result, the structure will still occupy $O(n)$ space, as only one node is needed for each person.

\section{Problem 5}

\begin{enumerate}[label=\alph*.]
    \item The historical score counts can be stored in a hash table.
        Each element contains the $(i,j)$ pair and a count of how many historical games had that score.
        Both $i$ and $j$ will be used by the hash function. The resulting value will then be used in conjunction
        with a modulo to find the position in an array, making the hash table.
        Therefore, populating the structure will take $O(n)$ time. Querying would run in constant time.
    \item Both the hashing function and getting the element at the hash index in the array takes constant time.
        Therefore adding, updating, and querying an entry in hash table will all take constant time.
        While processing the data, these constant time operations will yield an overall $O(n)$ computation time.
        A linked list can be used in each location of the array to deal with potential collisions.
        However, if the hash function is sufficiently uniform and random, this is unlikely.
        It is also important that the table should be of a size larger than the number of game score combinations.
        During half-time, the query operation will hash the score and find the sum of games.
        This will once again take constant time.
\end{enumerate}

\section{Problem 6}
\begin{enumerate}[label=\alph*.]
    \item The algorithm is written in psuedocode in \textbf{Algorithm \ref{mergesort3}}
    \begin{algorithm}
        \caption{Three-split merge sort}
        \label{mergesort3}
        \begin{algorithmic}[1] % The number tells where the line numbering should start
            \Procedure{Merge}{$A,B,C$} \Comment{Merge three sorted arrays}
                \State $final \gets []$ \Comment{Create empty array}
                \While{$len(A)+len(B)+len(C)>0$}
                    \If{$next(A)<next(B) \land next(A)<next(C)$}
                        \State $value \gets pop(A)$ \Comment{$A$ has next smallest value}
                    \ElsIf{$next(B)<next(A) \land next(B)<next(C)$}
                        \State $value \gets pop(B)$
                    \Else \Comment{must be equal or $next(C)<next(A) \land next(C)<next(B)$}
                        \State $value \gets pop(C)$
                    \EndIf

                    % Continue HERE
                    \State $append(final, value)$
                \EndWhile
            \EndProcedure
            \item[]
            \Procedure{Mergesort3}{$A$} \Comment{The sorted form of A}
                \If{$len(A) > 0$} \Comment{Only continue if array has elements}
                    \State $leftIndex \gets \lfloor\frac{len(A)}{3}\rfloor$
                    \State $rightIndex \gets \lceil\frac{2 \cdot len(A)}{3}\rceil$
                    \item[]
                    \State $left \gets \Call{Split}{A, 0, leftIndex}$
                    \State $middle \gets \Call{Split}{A, leftIndex, rightIndex}$
                    \State $right \gets \Call{Split}{A, rightIndex, len(A)}$
                    \item[]
                    \State $left \gets \Call{Mergesort3}{left}$ \Comment{Operate on subarray}
                    \State $middle \gets \Call{Mergesort3}{middle}$
                    \State $right \gets \Call{Mergesort3}{right}$
                    \State $sorted \gets \Call{Merge}{left, middle, right}$ \Comment{Merge sorted arrays}
                    \item[]
                    \State \textbf{return} $sorted$\Comment{The array is sorted}
                \EndIf
                \State \textbf{return} $A$\Comment{The empty array is sorted}
            \EndProcedure
        \end{algorithmic}
    \end{algorithm}
    \item The runtime of this new merge sort is $O(n\log_3 n)$ = $O(n\log n)$.
        This is because the merge operation takes $O(n)$ time, operating $log_3 n$ times.
        This is because each split operation divides each subarray into a third of the size.
        Therefore, the recursion tree will be $\log_3 n$ in depth, with even splitting.
    \item This algorithm is not faster than the standard merge sort algorithm,
        as the runtime is the same order.
        As the only difference is the log base, there is a constant factor between the original and new algorithm.
\end{enumerate}

\end{document}
