\documentclass{article}
\usepackage[utf8]{inputenc}
\usepackage{amsmath}
\usepackage{amssymb,latexsym,amsmath,epsfig,amsthm}


\title{Homework 1}
\author{Lucien Gaitskell}
\date{October 2020}

\begin{document}

\maketitle

\section{Problem 1}
\subsection{Approach}

\begin{enumerate}
    \item Evaluate Big-O of each individual function. Describe: How did you get there? Which parts are 'not important'?
    \item Break down into groups based on:
    \begin{itemize}
        \item logarithmic
        \item polylogarithmic
        \item square root
        \item linear
        \item linearithmic
        \item quadratic
        \item cubic
        \item exponential
    \end{itemize}
    \item Order within each group. Describe the method behind the ordering.
    \item Order the groups. Descrive the method for organizing groups and why the groups can be ordered together.
    \item Combine the groups into a list
\end{enumerate}

\subsection{List (with Big-O)}


\begin{align}
    6n\log n &= O(n\log n) \\
    2^{100} &= O(1) \\
    \log \log n &= O(\log \log n) \\
    \log^2 n &= O(\log^2 n) \\
    2^{2^{n}} &= O(4^{n}) \\
    \lceil \sqrt{n} \rceil &= O(n^{\frac{1}{2}}) \\
    n^{0.01} &= O(n^{\frac{1}{100}}) \\
    \frac{1}{n} &= O(n^{-1}) \\
    4n^{\frac{3}{2}} &= O(n^{\frac{3}{2}}) \\
    3n^{0.5} &= O(n^{\frac{1}{2}}) \\
    5n &= O(n)\\
    \lfloor 2n \log^2n \rfloor &= O(n \log^2n) \\
    2^n &= O(2^n) \\
    n \log_4 n &= O(n \log_4 n) \\
    4^n &= O(4^n) \\
    n^3 &= O(n^3) \\
    n^2 \log n &= O(n^2 \log n) \\
    \sqrt{\log n} &= O(\sqrt{\log n})
\end{align}

\subsection{Grouping}

\begin{align*}
    \intertext{\subsubsection*{Constant}}
    2^{100} &= O(1)
    \intertext{\subsubsection*{Polylogarithmic}}
    \sqrt{\log n} &= O(\log^{\frac{1}{2}}n) \\
    \log^2 n &= O(\log^2 n)
    \intertext{\subsubsection*{Inverse}}
    \frac{1}{n} &= O(n^{-1})
    \intertext{\subsubsection*{Root}}
    3n^{0.5} &= O(n^{\frac{1}{2}}) \\
    \lceil \sqrt{n} \rceil &= O(n^{\frac{1}{2}}) \\
    n^{0.01} &= O(n^{\frac{1}{100}})
    \intertext{\subsubsection*{Linearithmic}}
    n \log_4 n &= O(n \log_4 n) \\
    6n\log n &= O(n\log n)
    \intertext{\subsubsection*{Polynomial}}
    n^3 &= O(n^3) \\
    5n &= O(n) \\
    4n^{\frac{3}{2}} &= O(n^{\frac{3}{2}})
    \intertext{\subsubsection*{Exponential}}
    4^n &= O(4^n) \\
    2^n &= O(2^n) \\
    2^{2^{n}} &= O(4^{n})
    \intertext{\subsubsection*{Other}}
    \lfloor 2n \log^2n \rfloor &= O(n \log^2n) \\
    \log \log n &= O(\log \log n) \\
    n^2 \log n &= O(n^2 \log n)
\end{align*}

\subsection{Sorting Groups}

From smallest to largest:
\begin{enumerate}
    \item Constant
    \item Inverse
    \item Logarithmic
    \item Root
    \item Polynomial
    \item Exponential
\end{enumerate}

\vspace{5mm}
The other equations have to be sorted in-between.

\subsection{Final List}
\setcounter{equation}{0}
\begin{align}
    2^{100} &= O(1) \\
    \frac{1}{n} &= O(n^{-1}) \\
    \log \log n &= O(\log \log n) \\
    \sqrt{\log n} &= O(\log^{\frac{1}{2}}n) \\
    \log^2 n &= O(\log^2 n) \\
    6n\log n &= O(n\log n) \\
    n \log_4 n &= O(n \log_4 n) \\
    \lfloor 2n \log^2n \rfloor &= O(n \log^2n) \\
    n^2 \log n &= O(n^2 \log n) \\
    n^{0.01} &= O(n^{\frac{1}{100}}) \\
    3n^{0.5} &= O(n^{\frac{1}{2}}) \\
    \lceil \sqrt{n} \rceil &= O(n^{\frac{1}{2}}) \\
    5n &= O(n) \\
    4n^{\frac{3}{2}} &= O(n^{\frac{3}{2}}) \\
    n^3 &= O(n^3) \\
    2^n &= O(2^n) \\
    4^n &= O(4^n) \\
    2^{2^{n}} &= O(4^{n})
\end{align}

\section{Problem 2}
\subsection{Approach}

\subsubsection{Question 1}
\begin{enumerate}
    \item Look at each loop
    \item Determine number of steps within each loop
    \item If a step in the loop is another loop, then determine for that loop and then include it as a step in the original loop
    \item Determine how many times the loop will execute
    \item Determine the number of remaining constant steps
\end{enumerate}

\subsubsection{Question 2}
\begin{enumerate}
    \item After determining T(n), setup an equation
    \item Solve for c and n or show that values exist
\end{enumerate}

\section{Question 3}
\subsection{Approach}
\begin{enumerate}
    \item "Log time sounds like trees"
    \item Could have a tree that holds central element
    \item Smaller elements to left, Greater elements to right
    \item Removing would cause bubble up
    \item Adding would cause bubble down
    \item May have to have null node in the case of even number of nodes: instead would take average of the two
\end{enumerate}


\end{document}
