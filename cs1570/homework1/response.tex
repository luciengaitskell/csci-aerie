\documentclass{article}
\usepackage[utf8]{inputenc}
\usepackage{amsmath}
\usepackage{amssymb,latexsym,amsmath,epsfig,amsthm}


\title{Homework 1}
\author{Lucien Gaitskell}
\date{October 2020}

\begin{document}

\maketitle

\section{Problem 1}

\subsection{List (with Big-O)}

Calculate Big-O of each given equation.

\begin{align}
    6n\log n &= O(n\log n) \\
    2^{100} &= O(1) \\
    \log \log n &= O(\log \log n) \\
    \log^2 n &= O(\log^2 n) \\
    2^{2^{n}} &= O(2^{2^{n}}) \\
    \lceil \sqrt{n} \rceil &= O(n^{\frac{1}{2}}) \\
    n^{0.01} &= O(n^{\frac{1}{100}}) \\
    \frac{1}{n} &= O(n^{-1}) \\
    4n^{\frac{3}{2}} &= O(n^{\frac{3}{2}}) \\
    3n^{0.5} &= O(n^{\frac{1}{2}}) \\
    5n &= O(n)\\
    \lfloor 2n \log^2n \rfloor &= O(n \log^2n) \\
    2^n &= O(2^n) \\
    n \log_4 n &= O(n \log n) \\
    4^n &= O(4^n) \\
    n^3 &= O(n^3) \\
    n^2 \log n &= O(n^2 \log n) \\
    \sqrt{\log n} &= O(\sqrt{\log n})
\end{align}

\subsection{Growth Rate Classifications}

Group each equation by their Big-O classifications.

\begin{center}
    \small From smallest to largest.
\end{center}

\begin{align*}
    \intertext{\subsubsection*{Inverse}}
    \frac{1}{n} &= O(n^{-1})
    \intertext{\subsubsection*{Constant}}
    2^{100} &= O(1)
    \intertext{\subsubsection*{Polylogarithmic}}
    \sqrt{\log n} &= O(\log^{\frac{1}{2}}n) \\
    \log^2 n &= O(\log^2 n)
    \intertext{\subsubsection*{Logarithmic}}
    \log \log n &= O(\log \log n)
    \intertext{\subsubsection*{Root}}
    3n^{0.5} &= O(n^{\frac{1}{2}}) \\
    \lceil \sqrt{n} \rceil &= O(n^{\frac{1}{2}}) \\
    n^{0.01} &= O(n^{\frac{1}{100}})
    \intertext{\subsubsection*{Linear}}
    5n &= O(n) \\
    \intertext{\subsubsection*{Linearithmic}}
    \lfloor 2n \log^2n \rfloor &= O(n \log^2n) \\
    n \log_4 n &= O(n \log n) \\
    6n\log n &= O(n\log n)
    \intertext{\subsubsection*{Quadratic}}
    4n^{\frac{3}{2}} &= O(n^{\frac{3}{2}}) \\
    n^2 \log n &= O(n^2 \log n)
    \intertext{\subsubsection*{Cubic}}
    n^3 &= O(n^3) \\
    \intertext{\subsubsection*{Exponential}}
    4^n &= O(4^n) \\
    2^n &= O(2^n) \\
    2^{2^{n}} &= O(2^{2^{n}})
\end{align*}

\subsection{Final List}

Final organization of equations based on Big-O size.

\setcounter{equation}{0}
\begin{align}
    \frac{1}{n} &= O(n^{-1}) \\
    2^{100} &= O(1) \\
    \log \log n &= O(\log \log n) \\
    \sqrt{\log n} &= O(\log^{\frac{1}{2}}n) \\
    \log^2 n &= O(\log^2 n) \\
    n^{0.01} &= O(n^{\frac{1}{100}}) \\
    3n^{0.5} =
    \lceil \sqrt{n} \rceil &= O(n^{\frac{1}{2}}) \\
    5n &= O(n) \\
    6n\log n =
    n \log_4 n &= O(n \log n) \\
    \lfloor 2n \log^2n \rfloor &= O(n \log^2n) \\
    4n^{\frac{3}{2}} &= O(n^{\frac{3}{2}}) \\
    n^2 \log n &= O(n^2 \log n) \\
    n^3 &= O(n^3) \\
    2^n &= O(2^n) \\
    4^n &= O(4^n) \\
    2^{2^{n}} &= O(2^{2^{n}})
\end{align}

\section{Problem 2}
\subsection{Approach}

\subsubsection{Question 1}
\begin{enumerate}
    \item Look at each loop
    \item Determine number of steps within each loop
    \item If a step in the loop is another loop, then determine for that loop and then include it as a step in the original loop
    \item Determine how many times the loop will execute
    \item Determine the number of remaining constant steps
\end{enumerate}

\subsubsection{Question 2}
\begin{enumerate}
    \item After determining T(n), setup an equation
    \item Solve for c and n or show that values exist
\end{enumerate}

\section{Question 3}
\subsection{Approach}
\begin{enumerate}
    \item "Log time sounds like trees"
    \item Could have a tree that holds central element
    \item Smaller elements to left, Greater elements to right
    \item Removing would cause bubble up
    \item Adding would cause bubble down
    \item May have to have null node in the case of even number of nodes: instead would take average of the two
\end{enumerate}


\end{document}
