\documentclass{article}
\usepackage[utf8]{inputenc}
\usepackage{amsmath}
\usepackage{amssymb,latexsym,amsmath,epsfig,amsthm}
\usepackage{enumitem}
\usepackage{ulem}
\usepackage{algorithm}
\usepackage{algpseudocode}


\title{Homework 1}
\author{Lucien Gaitskell}
\date{October 2020}

\begin{document}

\maketitle

\section{Problem 1}

\begin{enumerate}[label=(\alph*)]
    \item The algorithm is written in psuedocode in \textbf{Algorithm \ref{differencealg}}
    \begin{algorithm}
        \caption{Maximum difference of $n \times n$ matrix }
        \label{differencealg}
        \begin{tabular}{r l}
            \hspace*{\algorithmicindent} \textbf{Input}: & $A = n \times n$ table of numbers \\
            \hspace*{\algorithmicindent} \textbf{Output}:& $d=$ maximum difference in $A$ \\
                                                         %& $l=$ minimum value in $A[1..n-1, 1..n-1]$
        \end{tabular}
        
        \begin{algorithmic}[1] % The number tells where the line numbering should start
            \Function{Maximize}{$A$} \Comment{given $A$ and $n \times n$ matrix}
                \If{$n == 1$} \Comment{base case}
                    \State \Return $0, A[1,1]$
                    \item[] \Comment{difference of element to itself and smallest value ($A[1,1]$)}
                \EndIf
                \item[]
                
                \State $d, l \gets \Call{Maximize}{A[1..n-1, 1..n-1]}$ \Comment{get data for sub-matrix}
                \State $l_h \gets l$ \Comment{Create minimum value var for new vertical values}
                \State $l_v \gets l$ \Comment{Create minimum value var for new horizontal values}

                \item[]

                \For{$i = 1\to n, j = n-1$} \Comment{Horizontal operations}
                    \State $l_h \gets min(A[i,j], l_h)$ \Comment{Save smallest value for horizontal}
                    \State $d \gets max(A[i,j] - l_h, d)$ \Comment{Save maximum difference}
                \EndFor
                \item[]
                \For{$j = 1\to n-1, i = n$} \Comment{Vertical operations}
                    \State $l_v \gets min(A[i,j], l_v)$ \Comment{Save smallest value for vertical}
                    \State $d \gets max(A[i,j] - l_v, d)$ \Comment{Save maximum difference}
                \EndFor

                \item[]
                \State $l \gets min(l_v, l_h, A[n,n])$ \Comment{Save smallest value for for corner $(n,n)$}
                \State $d \gets max(A[n,n] - l, d)$ \Comment{Save maximum difference}

                \State \Return $d, l$ \Comment{maximum difference and mininum value}
            \EndFunction
            \item[]
            \Function{DifferenceMaximize}{$A$} \Comment{The sorted form of A}
                \State $d, l \gets \Call{Maximize}{A}$ \Comment{Run maximize function}
                \State \Return $d$  \Comment{Return only desired value}
            \EndFunction
        \end{algorithmic}
    \end{algorithm}

    % This algo doesn't account for the possibility that optimal solution is A[i,j]-A[i, j-1]
    %   possibly would need to make recursion at a finer resolution

    \item Algorithm \ref{differencealg} visits every possible $A[i, j]$
        and compares it's difference with the smallest $A[c,d]$ in the subarray $A[1..i, 1..j]$.
        Therefore, this algorithm must give the optimal value of $A[i, j]-A[c, d]$.
    \item As algorithm \ref{differencealg} visits every possible $A[i, j]$ only once,
        it takes $O(n^2)$ time.
        Although structured as a recursion, each parent step relys on data from the subsequent step
        only computing for the missing new $A[i, j]$ on $A[1..n, 1..n]$ using existing data from $A[1..n-1, 1..n-1]$.
\end{enumerate}

\end{document}
