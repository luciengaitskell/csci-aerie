\documentclass{article}
\usepackage[utf8]{inputenc}
\usepackage{amsmath}
\usepackage{amssymb,latexsym,amsmath,epsfig,amsthm}
\usepackage{enumitem}
\usepackage{algorithm}
\usepackage{algpseudocode}


\title{Midterm 1}
\author{Lucien Gaitskell}  % not supposed to in actual submission 
\date{January 2021}

\begin{document}

\maketitle

\section{Problem 1}

\begin{enumerate}[label=(\alph*)]
    \item The algorithm is written in psuedocode in \textbf{Algorithm \ref{differencealg}}
    \begin{algorithm}
        \caption{Maximum difference of $n \times n$ matrix }
        \label{differencealg}
        \begin{tabular}{r l}
            \hspace*{\algorithmicindent} \textbf{Input}: & $A = n \times n$ table of numbers \\
            \hspace*{\algorithmicindent} \textbf{Output}:& $d=$ maximum difference in $A$ \\
                                                         %& $l=$ minimum value in $A[1..n-1, 1..n-1]$
        \end{tabular}
        
        \begin{algorithmic}[1] % The number tells where the line numbering should start
            \Function{Maximize}{$A$} \Comment{given $A$ and $n \times n$ matrix}
                \If{$n == 1$} \Comment{base case}
                    \State \Return $0, A[1,1]$
                    \item[] \Comment{difference of element to itself and smallest value ($A[1,1]$)}
                \EndIf
                \item[]
                
                \State $d, l \gets \Call{Maximize}{A[1..n-1, 1..n-1]}$ \Comment{get data for sub-matrix}
                \State $l_h \gets l$ \Comment{Create minimum value var for new vertical values}
                \State $l_v \gets l$ \Comment{Create minimum value var for new horizontal values}

                \item[]

                \For{$i = 1\to n, j = n-1$} \Comment{Horizontal operations}
                    \State $l_h \gets min(A[i,j], l_h)$ \Comment{Save smallest value for horizontal}
                    \State $d \gets max(A[i,j] - l_h, d)$ \Comment{Save maximum difference}
                \EndFor
                \item[]
                \For{$j = 1\to n-1, i = n$} \Comment{Vertical operations}
                    \State $l_v \gets min(A[i,j], l_v)$ \Comment{Save smallest value for vertical}
                    \State $d \gets max(A[i,j] - l_v, d)$ \Comment{Save maximum difference}
                \EndFor

                \item[]
                \State $l \gets min(l_v, l_h, A[n,n])$ \Comment{Save smallest value for for corner $(n,n)$}
                \State $d \gets max(A[n,n] - l, d)$ \Comment{Save maximum difference}

                \State \Return $d, l$ \Comment{maximum difference and mininum value}
            \EndFunction
            \item[]
            \Function{DifferenceMaximize}{$A$} \Comment{The sorted form of A}
                \State $d, l \gets \Call{Maximize}{A}$ \Comment{Run maximize function}
                \State \Return $d$  \Comment{Return only desired value}
            \EndFunction
        \end{algorithmic}
    \end{algorithm}

    % This algo doesn't account for the possibility that optimal solution is A[i,j]-A[i, j-1]
    %   possibly would need to make recursion at a finer resolution

    \item Algorithm \ref{differencealg} visits every possible $A[i, j]$
        and compares it's difference with the smallest $A[c,d]$ in the subarray $A[1..i, 1..j]$.
        Therefore, this algorithm must give the optimal value of $A[i, j]-A[c, d]$.
    \item As algorithm \ref{differencealg} visits every possible $A[i, j]$ only once,
        it takes $O(n^2)$ time.
        Although structured as a recursion, each parent step relys on data from the subsequent step
        only computing for the missing new $A[i, j]$ on $A[1..n, 1..n]$ using existing data from $A[1..n-1, 1..n-1]$.
\end{enumerate}

\section{Problem 2}

\begin{enumerate}[label=(\alph*)]
    \item The algorithm is written in psuedocode in \textbf{Algorithm \ref{majorityalg}}
    \begin{algorithm}
        \caption{Majority element in an array}
        \label{majorityalg}

        \begin{tabular}{r l}
            \hspace*{\algorithmicindent} \textbf{Input}:  & $A[1..n] = n$ length array of non-ordered elements \\
            \hspace*{\algorithmicindent} \textbf{Output}: & $d=$ majority element value in $A$ or $\varnothing$ if none \\
        \end{tabular}

        \begin{algorithmic}[1]
            \Function{MajorityValue}{$A$} \Comment{$A$ an array of length $n$}
                \If{$n==0$}
                    \State \Return $\varnothing$
                \EndIf
                \If{$n==1$}
                    \State \Return $A[1]$
                \EndIf
                \item[]
                \State $v \gets A[1]$ \Comment{Selected value from array}
                \State $c \gets 0$ \Comment{Counter for local occurrence of $v$}
                \item[]
                \For{$v_i$ in $A[2..n]$}
                    \If{$v_i = v$} \Comment{Iterate counter if values match}
                        \State $c \gets c+1$
                    \Else
                        \If{$c > 0$} \Comment{If multiple prior occurrences only decrement}
                            \State $c \gets c-1$
                        \Else \Comment{$c=0, set new value$}
                            \State $v=v_i$
                        \EndIf
                    \EndIf
                \EndFor
                \item[]
                \State $c_g \gets 0$ \Comment{Counter for global occurrences of $v$}
                \For{$v_i$ in $A[1..n]$}
                    \If{$v=v_i$}
                        \State $c_g \gets c_g + 1$
                    \EndIf
                \EndFor
                \item[]
                \If{$c_g > \lfloor n/2 \rfloor$} \Comment{Return if element is majority}
                    \State \Return $v$
                \Else \Comment{There is no majority}
                    \State \Return $\varnothing$
                \EndIf
            \EndFunction
        \end{algorithmic}
    \end{algorithm}

    \item \textbf{Proof:}
    \begin{itemize}
        \item Case 1: \emph{$A$ has no majority}\\
        Therefore, regardless of $v$, the selected value, the global occurrences count
        from the second loop will never exceed $\lfloor n/2 \rfloor$, by the definition
        of majority element.

        \item Case 2: \emph{$A$ has a majority} \\
        Given each value $a_i$ of $u$ unique elements in $A$, there are corresponding occurrence
        counts of $c_i$. Let the majority element value be $a_m$ and count be $c_m$.
        Therefore, $c_m > \sum_{i=1}^{m} c_i + \sum_{i=m}^{u} c_i$. \\
        In the worst case, $c_m = \lfloor n/2 \rfloor +1$, and each occurrence is interleaved
        between all other elements, elements of value $a_m$ must start and finish the array.
        Therefore, the algorithm will return $a_m$, correctly.        
    \end{itemize}

    \item \textbf{Algorithm \ref{majorityalg}} operates over the full length $n$ of $A$ twice, serially.
    Each operation within the loops take $O(1)$ constant time.
    Therefore the runtime is $O(n) + O(n) = O(n)$.
\end{enumerate}

\section{Problem 3}
\begin{enumerate}[label=(\alph*)]
    \item The algorithm is written in psuedocode in \textbf{Algorithm \ref{hotelalg}}
    \begin{algorithm}
        \caption{Optimal hotel path for Bilbo}
        \label{hotelalg}

        \begin{tabular}{r l}
            \hspace*{\algorithmicindent} \textbf{Inputs}:    & $D[1..n] = n$ length array of distances to next hotel. \\
                                                            & $R[1..n] = n$ length array of ranges from each hotel \\
                                                            & ($D[n] = R[n] = \varnothing$ represents the end) \\
            \hspace*{\algorithmicindent} \textbf{Output}:   & $P[1..m] = m$ length array of the path containing \\
                                                            & each hotel index from A in the path \\
        \end{tabular}

        \begin{algorithmic}[1]
            \Function{HotelPath}{$D[1..n], R[1..n]$}
                \State $i_c = 1$            \Comment{Set current hotel index to first}
                \State $P \gets [i_c]$ \Comment{Create path starting at current hotel}
                \item[]
                \While{$i_c < n$}           \Comment{Continue until at end of journey}
                    \If{$D[i_c] > R[i_c]$}        \Comment{Path impossible if next hotel exceeds range}
                        \State \Return $\varnothing$
                    \Else                   \Comment{Default choosing subsequent hotel}
                        \State $i_{next} \gets i_c + 1]$
                        \State $t_{next} \gets D[i_c] + D[i_{next}]$    \Comment{Total range from hotel $i_c$}
                        \State $d_t \gets D[i_c] + D[i_{next}]$     \Comment{Distance counter}
                    \EndIf
                    \item[]
                    \State $i_s = i_c+2$    \Comment{Select next available hotel}
                    \While{$R[i_c] \geq d_t$ and $i_s \leq n$}  \Comment{Operate until past list or range}
                        \If{$R[i_s]==D[i_s]==\varnothing$}      \Comment{If possible, go to end}
                            \State $i_{next} = i_s$
                            \State $t_{next} = t_s$
                        \Else   \Comment{Select new hotel and test cumulative range}
                            \State $t_s \gets d_t + R[i_s]$
                            \If{$t_{next} < t_s$}   \Comment{Save if cumulative range is larger}
                                \State $i_{next} = i_s$
                                \State $t_{next} = t_s$
                            \EndIf
                        \EndIf
                        \item[]
                        \State $i_s \gets i_s + 1$  \Comment{Select next hotel to test}
                    \EndWhile
                    \item[]
                    \State $P \gets P + i_{next}$   \Comment{Save next hotel to path}
                    \State $i_c = i_{next}$         \Comment{Set current hotel to prior next hotel}
                \EndWhile
                \item[]
                \State \Return $P$  \Comment{Greedily selected optimal path}
            \EndFunction
        \end{algorithmic}
    \end{algorithm}

    \item The optimal solution to the problem will visit the fewest hotels,
        and therefore allow Bilbo to make his journey in the least time.
        This is a great use case for a greedy algorithm.
        Given a starting hotel $h_a$, there is a set of hotels $A$
        within range of $h_a$. Each hotel $h_i$ in $A$ has a distance from
        $h_a$ of $d_i$. The range from $h_i$ is $r_i$. The maximum potential
        range of $h_i$ is $t_i = d_i + r_i$.

        Repeatidly selecting the hotel $h_o$ with the maximum $t_o$ will allow
        Bilbo to travel the most distance every day.
        Therefore, on average, Bilbo will travel farther per day and will
        take an optimal path.

        The algorithm will return a null path if at one point no subsequent
        hotel could be reached. In this case the path is impossible.

    \item The algorithm operates using two loops. However, the outer
        loop only iterates once for every hotel $h \in P$,
        given $P$ the set of selected hotels in Bilbos path.
        The inner loop iterates once for every hotel $h \notin P$.
        All other operations within the loops take constant $O(1)$ time.
        These two sets encompasses the set of all hotels
        $H = (h \in P) \cup (h \notin P)$, of length $n$.
        Therefore the runtime is $O(n) \cdot O(1) = O(n)$.
\end{enumerate}

\pagebreak

\section{Problem 4}

\begin{enumerate}[label=(\alph*)]
    \item \textbf{Data structures:}
    \begin{enumerate}[label=(\arabic*)]
        \item \textbf{Frequent Fliers} are organized using a balanced search tree.
        The nodes in the tree are organized by the unique flier ID,
        with smaller IDs on the left and larger on the right.
        The node contains the flier ID and their frequent flier status.

        \textbf{Let this structure be named $P$.}

        \item \textbf{Flights} are organized using a balanced search tree.
        The nodes in the tree are organized by the unique flight ID,
        with smaller IDs on the left and larger on the right.
        The node contains the flight ID and a reference to an associated
        flight specific upgrade structures in point \ref{flightupgrade}.

        \textbf{Let this structure be named $F$.}

        \item \textbf{Flight Specific Upgrades} \label{flightupgrade} are organized
        using a heap priority queue.
        Each node represents a flier with an upgrade on the flight represented
        by the containing priority queue.
        Each node will also contain the date of submission of the upgrade request
        and references to the parent and both children nodes.
        The priority of each flier node is based on their frequent flier status,
        or time since request if a tie-break is necessary.
        
        There will also be a second structure associated with each flight:
        a balanced search tree containing each flier with an upgrade.
        The nodes will be sorted by the flier ID, and each will contain a reference
        to the associated flier node in the priority queue structure.

        \textbf{Let the priority structure be named $U_i$ for $i$ the flight ID (upgrades).}
        \textbf{Let the flier ID structure be named $W_i$ for $i$ the flight ID (waitlist).}
    \end{enumerate}

    \item The operations for \emph{Priority}, \emph{Upgrade request},
    \emph{Upgrade cancellation}, \emph{Priority update}, and \emph{Upgrade processing}
    are written in psuedocode in \textbf{Algorithms
    \ref{upgraderequest}, \ref{upgradecancel}, \ref{priorityupdate}, \ref{upgradeprocessing}}
    with helper functions in \textbf{Algorithm \ref{flierhelper}}.

    \begin{algorithm}
        \caption{Frequent flier helper operations}
        \label{flierhelper}

        \begin{algorithmic}[1]
            \Function{FlierStatus}{$x$}    \Comment{Get frequent flier status of flier $x$}
                \State $n \gets $ root node of $P$.
                \While{$n.key \not= x$} \Comment{Traverse tree until node is found}
                    \If{$x < n.key$}
                        \State $n = n.left$
                    \Else
                        \State $n = n.right$
                    \EndIf
                \EndWhile
                %\item[]
                \State \Return $n.status$   \Comment{Return selected flier status}
            \EndFunction
            \item[]\item[]
            \Function{FlierUpgradePriority}{$u_1, u_2$}
                \item[]\Comment{Determine if flier upgrade $u_1$ is higher priority than $u_2$}
                \State $s_1, t_1 \gets u_1$ \Comment{Unpack flier status and upgrade date}
                \State $s_2, t_2 \gets u_2$
                \item[]
                \If{$s_1 > s_2$}    \Comment{If flier $1$ has a higher status}
                    \State \Return $True$
                \ElsIf{$s_1 = s_2$}
                    \If{$t_1 < t_2$}    \Comment{If upgrade $1$ is older}
                        \State \Return $True$
                    \Else
                        \State \Return $False$
                    \EndIf
                \Else
                    \State \Return $False$
                \EndIf
            \EndFunction
            \item[]\item[]
            \Function{FlightQueue}{$f$} \Comment{Get upgrade queue of flight $f$}
                \State $n \gets $ root node of $F$.
                \While{$n.key \not= f$} \Comment{Traverse tree until node is found}
                    \If{$f < n.key$}
                        \State $n = n.left$
                    \Else
                        \State $n = n.right$
                    \EndIf
                \EndWhile
                \item[] \Comment{update priority structure and flier ID structure}
                \State \Return $(n.upgrades, n.waitlist)$
            \EndFunction
        \end{algorithmic}
    \end{algorithm}

    \begin{algorithm}
        \caption{Upgrade request}
        \label{upgraderequest}

        \begin{algorithmic}[1]
            \Function{UpgradeRequest}{$x, f$}   \Comment{Flier $x$ upgrade request on flight $f$}
                \State $s_x \gets \Call{FlierStatus}{x}$
                \State $t_x \gets$ current time
                \State $U_f, W_f \gets \Call{FlightQueue}{f}$
                \item[] \Comment{Add flier to end of priority queue}
                \State $n_{depth} \gets \lfloor\log_2{|U_f|}\rfloor$
                \State $n_{horiz} \gets |U_f|-2^{n_{depth}} + 1$
                \State $n \gets $ new node at depth $n_{depth}$ and horizontal $n_{horiz}$
                \State $n.priority = (s_x, t_x)$
                \State $n.flierID = x$
                \item[] \Comment{Restore heap}
                \While{$n$ is not root and $\Call{FlierUpgradePriority}{(s_x, t_x), n.parent}$}
                    \item[]\Comment{Traverse up tree until in a correct spot for priority}
                    \State $n_s \gets n.parent$
                    \State Replace parent node with $n$.
                    \State Replace original position of $n$ with $n_s$.
                \EndWhile
                \item[]
                \State $w \gets $ root node of $W_f$.
                \While{$w$ is not a insertion node} \Comment{Traverse tree until open spot}
                    \If{$x < w.key$}
                        \State $w = w.left$
                    \Else
                        \State $w = w.right$
                    \EndIf
                \EndWhile
                \item[]
                \State create a node at $w$
                \State $w.key = x$
                \State $w.ref = n$

            \EndFunction
        \end{algorithmic}

    \end{algorithm}

    \begin{algorithm}
        \caption{Upgrade cancel}
        \label{upgradecancel}
        \begin{algorithmic}[1]
            \Function{UpgradeCancel}{$x, f$}  \Comment{Flier $x$ upgrade cancel on flight $f$}
                \State $U_f, W_f \gets \Call{FlightQueue}{f}$
                \item[]
                \State $n \gets $ root node of $W_f$.
                \While{$n.key \not= x$} \Comment{Traverse tree until node is found}
                    \If{$f < n.key$}
                        \State $n = n.left$
                    \Else
                        \State $n = n.right$
                    \EndIf
                \EndWhile
                \State $x_{ref} = n.ref$
                \item[] \Comment{Delete node and restore tree shape}
                \State Traverse down tree until reaching a external node $w$ such that
                $w.key > w.left$ and $w.key > w.right$.  % this is kinda wack
                \State Swap node $w$ in place of $n$
                \item[] \Comment{Remove and adjust heap}
                \State Swap last node $u$ in sub-heap under $x_{ref}$ into the place of $x_{ref}$.
                \While{$u$ has a child of higher priority}
                    \State Swap $u$ with that child.
                \EndWhile
            \EndFunction
        \end{algorithmic}
    \end{algorithm}
    \begin{algorithm}
        \caption{Priority Update}
        \label{priorityupdate}
        \begin{algorithmic}[1]
            \Function{PriorityUpdate}{$x, f$} \Comment{Flier $x$ update request on flight $f$}
                \State $s_x \gets \Call{FlierStatus}{x}$
                
                \State $U_f, W_f \gets \Call{FlightQueue}{f}$
                \item[]
                \State $n \gets $ root node of $W_f$.
                \While{$n.key \not= x$} \Comment{Traverse tree until node is found}
                    \If{$f < n.key$}
                        \State $n = n.left$
                    \Else
                        \State $n = n.right$
                    \EndIf
                \EndWhile
                \State $x_{ref} = n.ref$    \Comment{Reference to flier node in priority queue}
                \State $t_x \gets n_{ref}.time$
                \item[]
                \State $x_{ref}.priority = (x_{ref}, time)$

                \While{$x_{ref} \not =$ root and $\Call{FlierUpgradePriority}{x_{ref}, x_{ref}.parent}$}
                    \item[]\Comment{Traverse up tree until in a correct spot for priority}
                    \State $n_s \gets n.parent$
                    \State Replace parent node with $n$.
                    \State Replace original position of $n$ with $n_s$.
                \EndWhile

                \State Ensure $n.ref$ points to the new location in the priority queue.
            \EndFunction
        \end{algorithmic}

    \end{algorithm}
    \begin{algorithm}
        \caption{Upgrade Processing}
        \label{upgradeprocessing}
        \begin{algorithmic}[1]
            \Function{UpgradeProcessing}{$f, k$}  \Comment{Flight $f$ upgrade given $k$ seats}
                \State $U_f, W_f \gets \Call{FlightQueue}{f}$
                \State $S \gets $ empty array of length $k$ \Comment{Selected users array}
                \For{$i$ from $1$ to $k$}
                    \State Append root node of $U_f$ to $S$
                    \item[]
                    \State Replace root node with last node $w$ in $U_f$
                    \While{$w.priority$ is not greater than both children}
                        \State Swap $w$ with a child that has a greater priority than $w$
                    \EndWhile
                \EndFor
                \item[]
                \State \Return $S$
            \EndFunction
        \end{algorithmic}

    \end{algorithm}

    \pagebreak
    \item \textbf{Data structures:}
    \begin{enumerate}[label=(\arabic*)]
        \item \textbf{Frequent Fliers ($P$)} \\
        One node is created for each flier. Therefore there will be $m$ nodes,
        for the $m$ frequent fliers. This yields a $O(m)$ space complexity.

        \item \textbf{Flights ($F$)} \\
        One node is create for each flight. Therefore there will be $n$ nodes,
        for the $n$ flights. This yields a $O(n)$ space complexity.

        \item \textbf{Flight Specific Upgrades ($U_i, W_i$)} \\
        One nodes will be created in both structures for each frequent flier
        per flight. Therefore there will be $2 n_f$ nodes for each flight $f$.
        In total there will be $O(\sum_{f\in F}{2*n_f})$
        Therefore, the structures will have $O(\sum_{f\in F}{n_f})$ space complexity.
    \end{enumerate}

    \textbf{Correctness:}
    \begin{enumerate}[label=(\arabic*)]
        \item \emph{Priority} - \Call{FlierUpgradePriority}{} \\
            Comparing priority of two known upgrade requests take $O(1)$ time,
            as only simple contant time comparisons are performed

        \item \emph{Upgrade request} - \Call{UpgradeRequest}{} \\
            First the flier status has to be determined.
            All the flights are structured in a binary search tree
            and therefore finding a specific flier takes $O(\log m)$

            The upgrade request next searches through the flights by ID
            to get the flight's queue. As the flights are structured in a
            binary search tree, this takes $O(\log n)$.

            Creating a new node in the priority queue takes $O(\log{height} = \log{n_f})$
            as the full hight has to be traversed.
            Next, restoring the heap will also take a maximum of $O(\log{n_f})$
            as the operation only goes up the tree.

            Finally, the flier ID has to be added to the flier ID
            sorted flight waitlist. The insert operation only operates
            down the tree and therefore takes $O(\log{height} = \log{n_f})$ time.

            Therefore in total the operation is a time complexity of: \\
            $O(\log m) + O(\log n) + O(\log{n_f}) + O(\log{n_f}) + O(\log{n_f})$ \\
            $=O(\log m + \log n + \log{n_f})$

        \item \emph{Upgrade cancellation} - \Call{UpgradeCancelz}{} \\
            Similar to \emph{Upgrade Request}, the flight has to be
            looked up by ID: $O(\log n)$.
            
            A particular flier in the upgrade waitlist has to be found,
            and has to be found in the binary search tree.
            This operations takes $O(\log n_f)$ time complexity.

            Subsequently, this node in the waitlist is removed,
            and the binary search tree has to be restored.
            This operation only runs up the tree and therefore runs in
            $O(\log n_f)$ time.

            The referenced node in the priority queue also has to be removed.
            The operation to restore the structure of the heap, only runs
            up the tree and therefore will take $O(\log n_f)$ time.
            
            Therefore the operation will be performed in: \\
            $O(\log n) + O(\log n_f) + O(\log n_f) + O(\log n_f)$ \\
            $=O(\log n + \log n_f)$

        \item \emph{Priority update} - \Call{PriorityUpdate}{} \\
            First the flier status has to be determined.
            This is still $O(\log m)$ time complexity.
            
            The flight also has to be looked up by ID: $O(\log n)$.

            A particular flier in the upgrade waitlist has to be found:
            $O(\log n_f)$

            The referenced node in the priority queue is then moved
            only up the tree until the correct spot is found.
            Therefore this operation runs in
            $O(\log height) = O(\log n_f)$ time.

            In total these operations take: \\
            $O(\log m) + O(\log n) + O(\log n_f) + O(\log n_f)$ \\
            = $O(\log m + \log n + \log n_f)$

        \item \emph{Upgrade processing}
            The flight has to be looked up by ID: $O(\log n)$.

            For the $k$ fliers to be upgraded,
            the top node is first removed representing the next
            flier to be upgraded. The top node is the top priority
            flier by the properties of a priority queue heap.
            Next, the heap has to be restored:

            First, the last node is swapped to the top,
            which takes $O(\log n_f)$ to reach as the node is at
            the bottom of the heap. 
            Next, the new top node is swapped only down the heap until
            it satisfies the heap rules, therefore taking $O(\log n_f)$
            These two operations occur $k$ times for each upgraded flier.

            In total these operations take: \\
            $O(\log n) + k\cdot (O(\log n_f) + O(\log n_f))$ \\
            $= O(\log n + k\log n_f)$

    \end{enumerate}

\end{enumerate}

\end{document}
